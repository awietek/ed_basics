\documentclass[]{article}

%These tell TeX which packages to use.
\usepackage{array,epsfig}
\usepackage{amsmath}
\usepackage{amsfonts}
\usepackage{amssymb}
\usepackage{amsxtra}
\usepackage{amsthm}
% \usepackage{mathrsfs}
\usepackage{color}
\usepackage{hyperref}

%Here I define some theorem styles and shortcut commands for symbols I use often
\theoremstyle{definition}
\newtheorem{defn}{Definition}
\newtheorem{thm}{Theorem}
\newtheorem{cor}{Corollary}
\newtheorem*{rmk}{Remark}
\newtheorem{lem}{Lemma}
\newtheorem*{joke}{Joke}
\newtheorem{ex}{Example}
\newtheorem*{soln}{Solution}
\newtheorem{prop}{Proposition}

\newcommand{\lra}{\longrightarrow}
\newcommand{\ra}{\rightarrow}
\newcommand{\surj}{\twoheadrightarrow}
\newcommand{\graph}{\mathrm{graph}}
\newcommand{\bb}[1]{\mathbb{#1}}
\newcommand{\Z}{\bb{Z}}
\newcommand{\Q}{\bb{Q}}
\newcommand{\R}{\bb{R}}
\newcommand{\C}{\bb{C}}
\newcommand{\N}{\bb{N}}
\newcommand{\M}{\mathbf{M}}
\newcommand{\m}{\mathbf{m}}
\newcommand{\MM}{\mathscr{M}}
\newcommand{\HH}{\mathscr{H}}
\newcommand{\Om}{\Omega}
\newcommand{\Ho}{\in\HH(\Om)}
\newcommand{\bd}{\partial}
\newcommand{\del}{\partial}
\newcommand{\bardel}{\overline\partial}
\newcommand{\textdf}[1]{\textbf{\textsf{#1}}\index{#1}}
\newcommand{\img}{\mathrm{img}}
\newcommand{\ip}[2]{\left\langle{#1},{#2}\right\rangle}
\newcommand{\inter}[1]{\mathrm{int}{#1}}
\newcommand{\exter}[1]{\mathrm{ext}{#1}}
\newcommand{\cl}[1]{\mathrm{cl}{#1}}
\newcommand{\ds}{\displaystyle}
\newcommand{\vol}{\mathrm{vol}}
\newcommand{\cnt}{\mathrm{ct}}
\newcommand{\osc}{\mathrm{osc}}
\newcommand{\LL}{\mathbf{L}}
\newcommand{\UU}{\mathbf{U}}
\newcommand{\support}{\mathrm{support}}
\newcommand{\AND}{\;\wedge\;}
\newcommand{\OR}{\;\vee\;}
\newcommand{\Oset}{\varnothing}
\newcommand{\st}{\ni}
\newcommand{\wh}{\widehat}

%Pagination stuff.
\setlength{\topmargin}{-.3 in}
\setlength{\oddsidemargin}{0in}
\setlength{\evensidemargin}{0in}
\setlength{\textheight}{9.in}
\setlength{\textwidth}{6.5in}
\pagestyle{empty}
\usepackage{braket}
\usepackage{natbib}
\usepackage{bibentry}
\bibliographystyle{plainnat}



\begin{document}

\begin{center}
  {FOR 1807 Winter School 2018 Marburg}\\
  \textbf{Exact Diagonalization hands-on
    session}\\ %You should put your name here
  Exercise sheet%You should write the date here.
\end{center}

\vspace{0.2 cm}

In the following exercises we will make use of the scripts
\texttt{hamiltonian\_tfi.py}, \texttt{hamiltonian\_hb\_staggered.py}
and \texttt{hamiltonian\_hb\_xxz.py}. Internally, spin states are
encoded in a binary representation, for example
\begin{equation}
  \ket{\uparrow\downarrow\uparrow \uparrow} = (1011)_2 = 11.
\end{equation}
The three files contain each contain a function
\texttt{get\_hamiltonian\_sparse} which creates the Hamiltonian matrix
in a sparse matrix format. The following Hamiltonians are given
\begin{itemize}
\item \texttt{hamiltonian\_tfi.py}: The \textit{Transverse Field
    Ising} model,
  \begin{equation}
    H = J\sum\limits_{\langle i , j \rangle} S^z_iS^z_j + h_x \sum\limits_{i}S^x,
  \end{equation}
  where $S^z$ and $S^x$ are the spin-1/2 spin operators, on a
  one-dimensional chain lattice (with periodic boundary conditions).
  This model does not conserve total $S^z$. Momentum conservation is
  not implemented.
\item \texttt{hamiltonian\_hb\_staggered.py}: The \textit{Heisenberg
    model in a staggered magnetic field} ,
  \begin{equation}
    \label{eq:hb_stag}
    H = J\sum\limits_{\langle i , j \rangle} \mathbf{S}_i\cdot\mathbf{S}_j + h_S \sum\limits_{i}(-1)^iS^z,
  \end{equation}
  one-dimensional chain lattice (with periodic boundary conditions).
  This model does conserve total $S^z$. Momentum conservation is not
  implemented.
\item \texttt{hamiltonian\_hb\_xxz.py}: The \textit{Heisenberg model
    with Ising anisotropy} ,
  \begin{equation}
    \label{eq:hb_xxz}
    H = J\sum\limits_{\langle i , j \rangle} \left[S^z_iS^z_j +
      (1+\Delta)\left(S^x_iS^x_j + S^y_iS^y_j\right)\right]
  \end{equation}
  one-dimensional chain lattice (with periodic boundary conditions).
  This model does conserve total $S^z$. Momentum conservation is
  implemented.
\end{itemize}
An example how to use these functions for computations can be found in
the file \texttt{example\_groundstate\_energy.py}
\newpage
\subsection*{Exercise 1: Impementing a Hamiltonian, computing static
  correlations (basic)}
\begin{enumerate}
\item Use the script \texttt{hamiltonian\_tfi.py} to compute ground
  state energies for $J$ and for a range of values $h_x$ from $0$ to
  $1$.
\item Alter the script \texttt{hamiltonian\_tfi.py} to create the
  Hamiltonian of the Heisenberg model in a staggered magnetic field,
  as given by Eq.~\ref{eq:hb_stag}. Cross check the ground state
  energy with the script \texttt{hamiltonian\_hb\_staggered.py} for
  total $S^z=0$.
\item Compute the ground state spin correlations
  $\braket{0|S^z_0S^z_r|0}$ and
  $\braket{0|\mathbf{S}_0\cdot\mathbf{S}_r|0}$ of the Heisenberg model
  in a staggered magnetic field. Check whether
  $$\braket{0|\mathbf{S}_0\cdot\mathbf{S}_r|0} = 3\braket{0|S^z_0S^z_r|0}$$
  for $h_s=0$.
\end{enumerate}

\subsection*{Exercise 2: Computing dynamical correlations functions}
In this exercise we compute the ground state Greeen's function,
\begin{equation}
  \label{eq:greensfunction}
  G^{zz}(\mathbf{q}, z) = \braket{0|S^{\dagger}(\mathbf{q}) 
    \frac{1}{z-H}S(\mathbf{q})|0}, \text{ where } z = \omega+i\eta,
\end{equation}
and the corresponding dynamical spin structure factor
\begin{equation}
  \label{eq:dynspin}
  S^{zz}(\mathbf{q}, \omega) = -\frac{1}{\pi}\lim_{\eta \rightarrow 0}\text{Im}
  G^{zz}(\mathbf{q}, \omega + i\eta),
\end{equation}
where
$$S(\mathbf{q}) = \frac{1}{\sqrt{L}}\sum\limits_{\mathbf{r}} e^{-i\mathbf{q}{\mathbf{r}}}.$$
We will compute these quantities for the isotropic Heisenberg chain,
as in Eq.~\ref{eq:hb_stag} with $h_s=0$.  $L$ denotes the length of
the chain whose discrete momenta are given by
$\mathbf{q} = 2\pi/L \cdot 0, 2\pi/L \cdot 1, \ldots, 2\pi/L \cdot
(L-1)$.  The Hamiltonian $H$ can be created using the script
\texttt{hamiltonian\_hb\_staggered.py}.
\begin{enumerate}
\item (basic) Fix a small but finite $\eta$. Compute the Green's
  function Eq.~\ref{eq:greensfunction} and the dynamical spin
  structure factor Eq.~\ref{eq:dynspin} by creating a dense matrix $H$
  of the Hamiltonian and inverting $(z - H)$ numerically for small
  system sizes.
\item (advanced) The Green's function can be evaluated without fully
  inverting the matrix $(z-H)$ using the \textit{continuous fraction
    expansion},
  \begin{equation}
    \label{eq:contfrac}
    G^{zz}(\mathbf{q}, z) =
    \frac{\braket{0|S^{\dagger}(\mathbf{q})S(\mathbf{q})|0}}
    {z - \alpha_0- \frac{\beta_1^2}{z - \alpha_1 - \frac{\beta_2^2}{z-\cdots}}}
  \end{equation}
  Here, $\alpha_n$, $\beta_n$ are given by the Lanczos recursion,
  \begin{align*}
    \ket{v_{n+1}} &= \beta_{n+1}^{-1}\left[(H-\alpha_n)\ket{v_n} - \beta_n \ket{v_{n-1}}\right] \\
    \alpha_n &= \braket{v_n|H|v_n} \\
    \beta_{n+1} &= \parallel(H-\alpha_n)\ket{v_n} - \beta_n \ket{v_{n-1}}\parallel,
  \end{align*}
  with starting vector $\ket{v_0} = S(\mathbf{q})\ket{0}$. A basic
  implementation of the Lanczos recursion creating the coefficients
  $\alpha_n$ and $\beta_n$ can be found in the file
  \texttt{algorithm\_lanczos.py}. Compare the results from continuous
  fraction expansion to the results from full inversion. For details on
  the continuous fraction expansion see e.g. \cite{koch20118}.
\end{enumerate}

\subsection*{Exercise 3: Hamiltonian symmetries}
In this exercise we investigate the Heisenberg XXZ model
Eq.~\ref{eq:hb_xxz}. This model has a discrete translational symmetry
$T:\mathbf{S}_i \mapsto \mathbf{S}_{i+1}$. Hence, a symmetrized basis
with fixed momentum $2\pi\mathbf{k}\L$ can be chosen. The script
\texttt{hamiltonian\_hb\_xxz.py} creates the Hamiltonian of the
Heisenberg model with Ising anisotropy in the basis of momentum
eigenstates. $\mathbf{k}$ can be specified.
\begin{enumerate}
\item (basic) Compute the low-lying eigenvalues of the Hamiltonian for
  each $\mathbf{k} = 0, \ldots, L-1$ and plot the energies as a
  function of momentum for the isotropic Heisenberg case.  Cross-check
  the eigenvalues by also computing them in a non-symmetrized basis by
  using the script \texttt{hamiltonian\_hb\_staggered.py}.
\item (expert) The model has an additional $C_2$ reflectional
  symmetry, $P:\mathbf{S}_i \mapsto \mathbf{S}_{L-1-i}$, which has an
  even and an odd representation. Extend the script
  \texttt{hamiltonian\_hb\_xxz.py} to also work in a symmetrized basis
  of both parity and translational symmetry. Compute the eigenvalues
  of the Hamiltonian in this new symmetrized basis and compare them
  with the eigenvalues without parity symmetry. For details on the
  implementation of discrete symmetries see e.g. \cite{weise}.
\end{enumerate}
\bibliography{exercises}
\end{document}


